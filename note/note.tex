\documentclass[reqno]{amsart}

\usepackage{amssymb}
\usepackage{bm}
\usepackage{graphicx,xcolor,xspace}

\newcommand{\recheck}[1]{{\color{red} #1}}
\newcommand{\rewrite}[1]{{\color{blue} #1}}

\newcommand{\bx}{\bm{x}}
\newcommand{\bv}{\bm{v}}
\newcommand{\bu}{\bm{u}}
\newcommand{\bU}{\bm{U}}
\newcommand{\bF}{\bm{F}}
\newcommand{\bW}{\bm{W}}
\newcommand{\bG}{\bm{G}}
\newcommand{\bH}{\bm{H}}
\newcommand{\bw}{\bm{w}}
\newcommand{\bP}{\bm{P}}
\newcommand{\bQ}{\bm{Q}}
\newcommand{\bA}{\bm{A}}
\newcommand{\bR}{\bm{R}}
\newcommand{\bq}{\bm{q}}
\newcommand{\bmm}{\bm{m}}
\newcommand{\bM}{\bm{M}}
\newcommand{\veps}{\varepsilon}


\newcommand*\diff{\mathop{}\!\mathrm{d}}
\newcommand*\Diff[1]{\mathop{}\!\mathrm{d^#1}}
\newcommand{\transpose}{^{\operatorname{T}}}

\begin{document}

\title{Projective Integration for Inelastic Boltzmann equation}

\maketitle

\section{Boltzmann Equation with Inelastic collision}

\subsection{The inelastic collision operator}

Assume two particles with velocities $v$ and $v_*$ are going to collide. During the collision, there is some loss of momentum in the impact direction $\omega \in S^{d-1}$ ($S^{d-1}$ is the unit sphere in $\mathbb{R}^d$), resulting in the post-collisional velocities $v'$ and $v_*'$. Let $e$ stand for the restitution coefficient or inelasticity parameter, then
\begin{equation} \label{IO}
(v'-v_*')\cdot \omega=-e[(v-v_*)\cdot \omega], \quad 0 \leq e \leq 1.
\end{equation}

Using (\ref{IO}), $v'$ and $v_*'$ can be represented as
\begin{equation}\label{omega}
\left\{
\begin{aligned}
v'&=v-\frac{1+e}{2}[(v-v_*)\cdot \omega ]\omega, \\
v_*'&=v_*+\frac{1+e}{2}[(v-v_*)\cdot \omega]\omega.
\end{aligned}\right.
\end{equation}
which is $\omega$-representation. One can easily verify the conservation of momentum and dissipation of energy:
\begin{equation} \label{momentum}
v'+v_*'=v+v_*; \quad |v'|^2+|v_*'|^2-|v|^2-|v_*|^2=-\frac{1-e^2}{2}[(v-v_*)\cdot \omega]^2\leq 0.
\end{equation}

For numerical purpose, it is convenient to consider $\sigma$-representation:
\begin{equation} \label{sigma}
\left\{
\begin{aligned}
v'&=\frac{v+v_*}{2}+\frac{1-e}{4}(v-v_*)+\frac{1+e}{4}|v-v_*|\sigma, \\
v_*'&=\frac{v+v_*}{2}- \frac{1-e}{4}(v-v_*)-\frac{1+e}{4}|v-v_*|\sigma,
\end{aligned}\right.
\end{equation}
where $\sigma$ is another unit vector on $S^{d-1}$ and is related to $\omega$ as
\begin{equation} \label{relation}
(g\cdot \omega)\omega=\frac{1}{2}(g-|g|\sigma).
\end{equation}
Then the collision has the following weak form:
\begin{align} \label{weak22}
\int_{\mathbb{R}^d} Q(f,f)(v)\,\varphi(v)\,\diff v &=\frac{1}{2} \int_{\mathbb{R}^d} \int_{\mathbb{R}^d} \int_{S^{d-1}} B_{\sigma}(|g|,\sigma\cdot \hat{g})ff_*\left(\varphi'+\varphi_*'-\varphi-\varphi_* \right)\,\diff\sigma \diff v \diff v_*,
\end{align}
where $g=v-v_*$, and $(v',v_*')$ are given by (\ref{sigma}). The collision kernel $B_{\sigma}$ may take various forms depending on the types of interactions. The most commonly used form is the variable hard sphere (VHS) model:
\begin{equation} \label{VHS}
B_{\sigma}(|g|,\sigma\cdot \hat{g})=C_{\lambda}|g|^{\lambda}, \quad 0\leq \lambda \leq 1,
\end{equation}
where $C_{\lambda}> 0$ is some constant. Two special cases are: Maxwell molecules ($\lambda=0$) and hard spheres ($\lambda=1$).

\subsection{The inelastic Boltzmann equation}

The inelastic Boltzmann equation reads:
\begin{equation}
  \partial_t f + v\cdot\nabla_x f = \frac{1}{\varepsilon} Q_\text{in}(f)
\end{equation}


\section{Projective Ingerator}

Consider an ODE system:
\begin{equation}
  \left\{
  \begin{aligned}
  \frac{du}{dt} &= g(u(t)), \quad t>0 \\
  u(0) &= u_0.  
  \end{aligned}
  \right.
\end{equation}
where the dynamic has a scale separation, that is, the eigenvalues of $\frac{dg}{du}$ are clustered into two groups separated by a gap.

\subsection{The scheme}
\begin{itemize}
  \item Inner integrator: forward Euler method with small time step $\delta t$:
        $$u^{k+1} = u^k + \delta t g(u^k).$$
  \item Extrapolate in time:
        $$ u^{n+1} = u^{n,K+1} + (\Delta t - (K+1)\delta t)\frac{u^{n,K+1} - u^{n,K}}{\delta t}.$$
\end{itemize}



\section{Numerical}

We consider the Riemann problem (Sod shock tube problem) as test.
\subsection{Example}

The initial data are given by the Maxwellian distributions computed from the macroscopic quantities

\begin{align}
  (\rho_l, u_l, T_l) &= (1, 0, 1), \quad &\text{if } 0\leq x\leq 0.5, \\
  (\rho_r, u_r, T_r) &= (0.125, 0, 0.25) \quad &\text{if }0.5<x\leq 1.
\end{align}

We solve the $1D_x\times 2D_v$ inelastic Boltzmann equation with different $\varepsilon$s and $e$s.

\subsubsection{$\varepsilon=10^{-2}$}

We test the PFE (projective forward euler) scheme compared with the explicite Euler scheme. $\Delta x=0.02$,  for PFE scheme $\Delta t=0.004$, $\delta t=10^{-3}$ and for Euler scheme $\Delta t = 0.002$. The macroscopic quantities and solution at $T=0.1$ are shown in Figure~1.

\begin{figure}
\centering
\includegraphics[width=\textwidth]{../src/data_figs/e0.2kn-2.pdf}
\caption{Solutions with $e=0.2$. PFE vs. Euler}
\end{figure}
\begin{figure}
  \centering
  \includegraphics[width=0.6\textwidth]{../src/data_figs/e0.2kn-2_f.pdf}
  \caption{Solution $f$ in $v$ when $e=0.2$.}
\end{figure}

\begin{figure}
  \centering
  \includegraphics[width=\textwidth]{../src/data_figs/e0.8kn-2.pdf}
  \caption{Solutions with $e=0.8$. PFE vs. Euler}
\end{figure}

\begin{figure}
\centering
\includegraphics[width=0.6\textwidth]{../src/data_figs/e0.8kn-2_f.pdf}
\caption{Solution $f$ in $v$ when $e=0.8$.}
\end{figure}

\subsubsection{$\varepsilon=10^{-4}$.}

We perfom the same numerical experiment as above. However, since $\Delta t$ of the Euler scheme is restricted by $\varepsilon$ which is very small in this case. We can only run the PFE with a bigger time step $\Delta t = 0.002$.
\begin{figure}
\centering
\includegraphics[width=\textwidth]{../src/data_figs/e1.0kn-4.pdf}
\caption{Solutions with $e=1.0$. PFE}
\end{figure}




\bibliography{ref}{}
\bibliographystyle{unsrt}

\end{document}


