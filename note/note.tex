\documentclass[reqno]{amsart}

\usepackage{amssymb}
\usepackage{bm}
\usepackage{graphicx,xcolor,xspace}

\newcommand{\recheck}[1]{{\color{red} #1}}
\newcommand{\rewrite}[1]{{\color{blue} #1}}

\newcommand{\bx}{\bm{x}}
\newcommand{\bv}{\bm{v}}
\newcommand{\bu}{\bm{u}}
\newcommand{\bU}{\bm{U}}
\newcommand{\bF}{\bm{F}}
\newcommand{\bW}{\bm{W}}
\newcommand{\bG}{\bm{G}}
\newcommand{\bH}{\bm{H}}
\newcommand{\bw}{\bm{w}}
\newcommand{\bP}{\bm{P}}
\newcommand{\bQ}{\bm{Q}}
\newcommand{\bA}{\bm{A}}
\newcommand{\bR}{\bm{R}}
\newcommand{\bq}{\bm{q}}
\newcommand{\bmm}{\bm{m}}
\newcommand{\bM}{\bm{M}}
\newcommand{\veps}{\varepsilon}


\newcommand*\diff{\mathop{}\!\mathrm{d}}
\newcommand*\Diff[1]{\mathop{}\!\mathrm{d^#1}}
\newcommand{\transpose}{^{\operatorname{T}}}

\begin{document}

\title{Projective Integration for Inelastic Boltzmann equation}

\begin{abstract}
This paper introduces a new moment system to solve kinetic equations based on machine learning.
To this end, a set of generalized moments are firstly constructed through an autoencoder to optimally characterize the velocity distribution. 
The moment system is then closed with the aim of best capturing the associated dynamics.
The reduced system is interpretable like the conventional PDEs and works independently of the numerical discretizations. 
In the one-dimension BGK model, the algorithm achieves a uniform accuracy in a wide range of regimes spanning from the hydrodynamic limit to free molecular flow. This makes the proposed method a new candidate for solving kinetic equations with a wide variation of mean free path.
\end{abstract}
\maketitle




{ }~\newline TODO
\begin{itemize}
\item related literature, asymptotic preserving, time relaxed Monte Carlo
\item a schematic plot
\item result of classical moment method
\item comparison of computational cost with original
\item discussion: fully nonlinear or perturbative
% \item further work: entropy condition, hyperbolicity, boundary conditions, start from Navier-Stokes
\end{itemize}


\section*{Acknowledgement}
This work is supported in part by ONR grant N00014-13-1-0338
and the Major Program of NNSFC under grant 91130005.


\bibliography{ref}{}
\bibliographystyle{unsrt}

\end{document}


